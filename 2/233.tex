% REV01: Sat 23 Apr 2022 17:30:00 WIB
% START: Thu 12 Aug 2004 00:00:00 WIB
% This was "An Example Document" by Leslie Lamport (January 21, 1994).
% Modified by C. Bin Kadal

\documentclass{article}
\title{PENETAPAN PRESIDEN REPUBLIK INDONESIA\\
	NOMOR 11 TAHUN 1963\\
	TENTANG\\
	PEMBERANTASAN KEGIATAN SUBVERSI}
\author{PRESIDEN REPUBLIK INDONESIA,}
\date{}
\usepackage[a4paper, total={7in, 9in}]{geometry}
\usepackage{paralist}
\usepackage{fancyhdr}
\pagestyle{fancy}
\fancyhf{}
% \fancyhead[LE,RO]{LERO}
% \fancyhead[RE,LO]{RELO}
% \fancyfoot[CE,CO]{CECO}
\fancyfoot[LE,RO]{VauLSMorg 01-20220423}
\setlength\parindent{0pt}
\pagenumbering{gobble}
% \renewcommand{\labelenumi}{(\arabic{enumi}) }
% \renewcommand{\labelenumii}{\alph{enumii}. }

\begin{document}
\maketitle

Menimbang:

\renewcommand{\labelenumi}{(\alph{enumi}). }
\begin{enumerate}
\item bahwa kegiatan subversi merupakan bahaya bagi keselamatan dan kehidupan Bangsa dan Negara
yang sedang berevolusi membentuk masyarakat Sosialis Indonesia;
\item bahwa guna pengamanan usaha-usaha mencapai tujuan revolusi perlu adanya peraturan tentang
pemberantasan kegiatan subversi tersebut;
\item bahwa pengaturan ini adalah dalam rangka pengamanan usaha mencapai tujuan revolusi, sehingga
dilakukan dengan Penetapan Presiden.
\end{enumerate}

Mengingat:
\\

Pasal IV Ketetapan Majelis Permusyawaratan Rakyat Sementara Nomor I/MPRS/1960 berhubungan
dengan pasal 10 Ketetapan Majelis Permusyawaratan Rakyat Sementara Nomor II/MPRS/1960.
\\

\hspace{\fill}MEMUTUSKAN:\hspace{\fill}
\vspace{10pt}

Menetapkan:
\\

PENETAPAN PRESIDEN TENTANG PEMBERANTASAN KEGIATAN SUBVERSI.

\section*{\centering{}BAB 1\\KEGIATAN SUBVERSI}

\subsection*{\centering{}Pasal 2}

\renewcommand{\labelenumi}{(\arabic{enumi}) }
\renewcommand{\labelenumii}{\alph{enumii}. }
\renewcommand{\labelenumiii}{\arabic{enumiii}. }
\begin{enumerate}
\item Dipersalahkan melakukan tindak pidana subversi:
\begin{enumerate}
\item  barang-siapa melakukan sesuatu perbuatan dengan maksud atau nyata-nyata dengan maksud
atau yang diketahuinya atau patut diketahuinya dapat:
\begin{enumerate}
\item memutar balikkan, merongrong atau menyelewengkan ideologi negara Pancasila atau
haluan negara, atau
\item menggulingkan, merusak atau merongrong kekuasaan negara atau kewibawaan
Pemerintah yang sah atau Aparatur Negara, atau
\item menyebarkan rasa permusuhan atau menimbulkan permusuhan, perpecahan,
pertentangan, kekacauan, kegoncangan atau kegelisahan diantara kalangan penduduk
atau masyarakat yang bersifat luas atau diantara Negara Republik Indonesia dengan
sesuatu Negara sahabat, atau mengganggu, menghambat atau mengacaukan bagi
industri, produksi, distribusi, perdagangan, koperasi atau pengangkutan yang
diselenggarakan oleh Pemerintah, atau berdasarkan keputusan Pemerintah, atau yang
mempunyai pengaruh luas terhadap hajat hidup rakyat;
\end{enumerate}
\item barang siapa melakukan sesuatu perbuatan atau kegiatan yang menyatakan simpati bagi
musuh Negara Republik Indonesia atau Negara yang sedang tidak bersahabat dengan Negara
Republik Indonesia.
\item barangsiapa melakukan pengrusakan atau penghancuran bangunan yang mempunyai fungsi
untuk kepentingan umum atau milik perseorangan atau badan yang dilakukan secara luas;
\item barangsiapa melakukan kegiatan mata-mata:
\item barangsiapa melakukan sabotasei.
\end{enumerate}
\item Dipersalahkan juga melakukan tindak pidana subversi barangsiapa memikat perbuat tersebut pada
ayat (1) tersebut diatas.
\end{enumerate}

\subsection*{\centering{}Pasal 2}

Yang dimaksud dengan kegiatan mata-mata ialah perbuatan melawan hukum untuk:

% \renewcommand{\labelenumi}{(\arabic{enumi}) }
% \renewcommand{\labelenumii}{\alph{enumii}. }
% \renewcommand{\labelenumiii}{\arabic{enumiii}. }
\renewcommand{\labelenumi}{\alph{enumi}. }
\begin{enumerate}
\item memiliki, menguasai atau memperoleh dengan maksud untuk meneruskannya langsung maupun tidak
langsung kepada Negara atau organisasi asing ataupun kepada organisasi atau kaum kontra
revolusioner, sesuatu peta, rancangan, gambar atau tulisan tentang bangunan-bangunan militer atau
rahasia militer ataupun keterangan tentang rahasia Pemerintah dalam bidang politik, diplomasi atau
ekonomi;
\item melakukan penyelidikan untuk musuh atau Negara lain tentang hal tersebut pada huruf a atau
menerima dalam pemondokan, menyembunyikan atau menolong seorang penyelidik musuh;
\item mengadakan, memudahkan atau menyebarkan propaganda untuk musuh atau negara lain yang
sedang dalam keadaan tidak bersahabat dengan Negara Republik Indonesia;
\item melakukan suatu usaha bertentangan kepentingan Negara sehingga, terhadap seseorang dapat
melakukan penyelidikan penuntutan, perampasan atau pembatasan kemerdekaan, penjatuhan pidana
atau tindakan lainnya oleh atau atas kekuasaan musuh;
\item memberikan kepada/atau menerima dari musuh atau Negara lain yang sedang dalam tidak
bersahabat dengan Negara Republik Indonesia atau pembantu-pembantu musuh atau Negara itu,
sesuatu barang atau uang, atau melakukan sesuatu perbuatan yang menguntungkan musuh atau
Negara itu atau pembantu-pembantunya, atau menyukarkan, merintangi atau menggagalkan sesuatu
tindakan terhadap musuh atau Negara itu atau pembantu-pembantunya.
\end{enumerate}

\subsection*{\centering{}Pasal 3}

Yang dimaksudkan dengan sabotase ialah perbuatan seseorang yang dengan maksud atau nyata-nyata
dengan maksud, atau yang mengetahuinya atau patut diketahuinya merusak, merintangi, menghambat,
merugikan atau mengadakan sesuatu yang sangat penting bagi usaha Pemerintah, mengenai:

\begin{enumerate}
\item bahan-bahan pokok keperluan hidup rakyat yang diimpor atau diusahakan oleh Pemerintah;
\item produksi, distribusi dan koperasi yang diawasi Pemerintah;
\item obyek-obyek dan proyek-proyek militer, industri, produksi dan perdagangan Negara:
\item proyek-proyek pembangunan semesta mengenai industri, produksi, distribusi dan perhubungan lalu
lintas;
\item instalasi-instalasi Negara;
\item perhubungan lalu lintas (darat, laut, udara, dan telekomunikasi).
\end{enumerate}

\section*{\centering{}BAB II\\PENYIDIKAN DAN PENUNTUTAN KEGIATAN SUBVERSI}

\subsection*{\centering{}Pasal 4}

Untuk keperluan penyidikan dan penuntutan kegiatan subversi, alat-alat kekuasaan Negara wajib
memberikan bantuan secukupnya.

\subsection*{\centering{}Pasal 5}

Penyidikan dan penuntutan kegiatan subversi dijalankan menurut ketentuan-ketentuan yang berlaku dengan
pimpinan dan petunjuk-petunjuk Jaksa Agung/Oditur Jenderal, sekedar tidak ditentukan lain dalam peraturan
ini.

\subsection*{\centering{}Pasal 6}

\renewcommand{\labelenumi}{(\arabic{enumi}) }
\renewcommand{\labelenumii}{\alph{enumii}. }
\renewcommand{\labelenumiii}{\arabic{enumiii}. }
% \renewcommand{\labelenumi}{\alph{enumi}. }
\begin{enumerate}
\item Guna keperluan penyidikan, tiap pegawai yang diserahi tugas-penyidikan dalam lingkungan
wewenangnya di mana saja dan pada setiap waktu, bila perlu dengan bantuan alat-alat kekuasaan
lain serta dengan menghindahkan ketentuan-ketentuan dalam ayat-ayat berikut, dapat memasuki
sesuatu tempat serta melakukan penggeledahan dan penyitaan barang-barang, termasuk surat-surat
yang mempunyai atau dapat disangka mempunyai sangkut-paut dengan kegiatan subversi.
\item Terkecuali dalam keadaan tertangkap tangan, jika tindakan dilakukan dalam sebuah bangunan, maka
pegawai yang dimaksud pada ayat (1) dengan disertai dua orang saksi harus terlebih dahulu
menunjukkan surat perintah penggeledahan atau penyitaan yang dikeluarkan oleh pejabat penyidik
yang berwenang.
\item Dari tindakan tersebut pada ayat (2) dalam waktu dua kali dua puluh empat jam dibuat berita-acara
yang memuat nama dan jabatan pegawai yang melakukan tindakan itu, nama saksi-saksi yang
menyertainya, cara melakukan penggeledahan serta
\end{enumerate}

\section*{\centering{}BAB III\\ZONK?!}

\subsection*{\centering{}Pasal 7}

ZONK!?

\subsection*{\centering{}Pasal 8}

ZONK!?

\subsection*{\centering{}Pasal 9}

ZONK!?

\subsection*{\centering{}Pasal 10}

\begin{enumerate}
\item Pemeriksaan perkara pidana subversi dalam tingkat pertama dimulai selambat-lambatnya dalam
waktu satu bulan setelah berkas perkara diterima dikepaniteraan.
Pemeriksaan dilakukan dan putusan dijatuhkan dalam waktu. sesingkat-singkatnya.
\item Dalam hal ada permohonan banding, maka berkas perkara disampaikan kepada pengadilan yang
memeriksa dalam tingkat banding dalam waktu dua puluh satu hari. Pengadilan dalam tingkat
banding menjatuhkan putusan selambat-lambatnya dalam satu bulan sesudah berkas diterima,
atau jika diadakan pemeriksaan tambahan yang tidak dilakukan oleh Pengadilan itu sendiri, satu
bulan mulai hari diterimanya kembali berkas perkara tersebut.
\item Terhadap putusan yang memuat pembebasan seluruhnya atau sebagian dapat diajukan
permohonan banding.
\end{enumerate}

\subsection*{\centering{}Pasal 11}

\begin{enumerate}
\item Apabila terdakwa setelah dua kali berturut-turut dipanggil secara sah tidak hadir disidang, maka
pengadilan berwenang mengadilinya diluar kehadirannya (in absensia). Dalam hal ini pemanggilan
hanya sah jika dilakukan dengan cara penempatan dua kali berturut-turut, tiap kali dalam sekurang-
kurangnya dua surat kabar-harian yang ditunjuk oleh Hakim.
\item Putusan pengadilan termaksud pada ayat (1) diberitahukan kepada terdakwa dengan cara yang
memuat nama pengadilan yang menjatuhkan putusan, tanggal dan nomor putusan serta amar
putusan dua kali berturut-turut, tiap kali dalam sekurang-kurangnya dua surat kabar hahan yang
ditunjuk oleh penuntut umum Oditur yang bersangkutan. Sehelai dari surat kabar yang memuat
pemberitahuan tersebut dimasukkan dalam berkas perkara.
\item Terhadap putusan yang dijatuhkan diluar kehadiran terdakwa dapat diajukan permohonan banding.
Bagi terdakwa yang memohon banding tenggang waktu mengajukan permohonan dihitung mulai
hari tanggal terakhir dari surat-surat kabar yang memuat pemberitaan tersebut.
\end{enumerate}

\subsection*{\centering{}Pasal 12}

\begin{enumerate}
\item Tiap orang yang diperiksa sebagai saksi atau ahli wajib memberikan keterangan tentang
pengetahuannya yang berhubungan dengan perkara yang sedang diperiksa.
\item Dengan tidak mengurangi ketentuan-ketentuan yang berlaku mengenai rahasia bank, maka kewajiban
termaksud pada ayat (1) berlaku juga bagi mereka yang biasanya pengetahuannya tentang sesuatu
harus dirahasiakan karena jabatan atau kedudukannya yang bersangkutan, kecuali bagi para petugas
agama dan dokter dalam lingkungan tugas masing-masing.
\item Dengan kata "Jaksa" yang tercantum di dalam pasal 3 ayat (2) Undang-undang Nomor 23 Prp tahun
1960 tentang Rahasia Bank (Lembaran-Negara tahun 1960 Nomor 71), khusus dalam rangka
pemberantasan kegiatan subversi ini, diartikan juga setiap pegawai penyidik, sedangkan kata "Jaksa
Agung" diartikan juga Menteri/Panglima Angkatan yang bersangkutan.
\end{enumerate}

\section*{\centering{}BAB IV\\ANCAMAN PIDANA}

\subsection*{\centering{}Pasal 13}

\begin{enumerate}
\item Barangsiapa melakukan tindak pidana subversi yang dimaksudkan dalam pasal 1 ayat (1) angka 1, 2,
3, 4 dan ayat (2) dipidana dengan pidana mati, pidana penjara seumur hidup atau pidana penjara
selama-lamanya 20 (dua puluh) tahun.
\item Barangsiapa melakukan tindak pidana subversi yang dimaksudkan dalam pasal 1 ayat (1) angka 5
dipidana dengan pidana mati, pidana penjara seumur hidup atau pidana penjara selama-lamanya 20
(dua puluh) tahun dan/atau denda setinggi-tingginya 30 (tiga puluh) juta rupiah.
\end{enumerate}

\subsection*{\centering{}Pasal 14}

Benda baik milik maupun bukan milik terpidana yang diperoleh dari atau digunakan sebagai alat melakukan
tindak pidana subversi dapat dirampas.

\subsection*{\centering{}Pasal 15}

Barangsiapa dengan sengaja tidak memenuhi kewajiban tersebut dalam pasal 12 ayat (1) dipidana dengan
pidana penjara selama-lamanya 5 (lima) tahun atau denda setinggi-tingginya 5 (lima) ratus ribu rupiah.

\subsection*{\centering{}Pasal 16}

Perbuatan-perbuatan tersebut dalam pasal-pasal 13 dan 15 adalah kejahatan.

\subsection*{\centering{}Pasal 17}

\begin{enumerate}
\item Jika suatu tindak pidana subversi dilakukan oleh atau atas nama suatu badan hukum, perseroan,
perserikatan orang, yayasan atau organisasi lainnya, maka tindakan peradilan dilakukan, baik
terhadap badan hukum, perseroan, perserikatan orang, yayasan atau organisasi lainnya itu, baik
terhadap mereka yang memberi perintah untuk melakukan tindak pidana tersebut atau yang bertindak
sebagai pemimpin dalam perbuatan itu, maupun terhadap kedua-duanya.
\item Suatu tindak pidana subversi dilakukan juga oleh atau atas nama suatu badan hukum, perseroan,
perserikatan orang, yayasan atau organisasi lainnya, jika tindakan itu dilakukan oleh orang-orang
yang baik berdasar hubungan kerja maupun berdasar hubungan lain, bertindak dalam lingkungan
badan hukum, perseroan, perserikatan orang, yayasan atau organisasi lainnya itu, tanpa mengingat
apakah orang-orang tersebut masing-masing tersendiri melakukan tindak pidana itu atau pada mereka
bersama ada unsur-unsur tindak pidana tersebut.
\item Jika tindakan peradilan dilakukan terhadap suatu badan hukum, perseroan, perserikatan orang,
yayasan atau organisasi lainnya, maka badan hukum, perseroan, perserikatan orang, yayasan atau
organisasi lainnya itu pada waktu penuntutan diwakili oleh seorang, pengurus atau, jika ada lebih dari
seorang pengurus, oleh salah seorang dari mereka itu.
Wakil dapat diwakili oleh orang lain. Hakim dapat memerintahkan supaya seorang pengurus
menghadap sendiri di pengadilan, dan dapat pula memerintahkan supaya pengurus itu dibawa
kemuka hakim.
\item Jika tindakan peradilan dilakukan terhadap suatu badan hukum, perseroan perserikatan orang,
yayasan atau organisasi lainnya, maka segala panggilan untuk menghadap dan segala penyerahan
surat-surat panggilan itu ditujukan kepada kepala pengurus atau ditempat tinggal kepala pengurus itu
atau ditempat pengurus. bersidang atau berkantor.
\end{enumerate}

\section*{\centering{}BAB V\\PELAKSANAAN PUTUSAN}

\subsection*{\centering{}Pasal 18}

\begin{enumerate}
\item Putusan pengadilan yang dijatuhkan dalam tindak pidana subversi dilaksanakan menurut ketentuan-
ketentuan yang berlaku, kecuali jika dalam peraturan ini ditentukan lain.
\item Putusan pengadilan yang tidak memuat pidana mati tidak tertunda karena permohonan grasi.
\end{enumerate}

\section*{\centering{}BAB VI\\PENUTUP}

\subsection*{\centering{}Pasal 19}

Ketentuan pasal 63 ayat (2) K.U.H.P. (Kitab Undang-undang Hukum Pidana) tidak berlaku terhadap, tindak
pidana yang disebut dalam peraturan ini.

\subsection*{\centering{}Pasal 20}

Penetapan Presiden ini mulai berlaku pada hari diundangkan.

Agar supaya setiap orang dapat mengetahuinya memerintahkan pengundangan Penetapan Presiden ini
dengan penempatan dalam Lembaran-Negara Republik Indonesia.
\\

Ditetapkan Di Jakarta\\
Pada Tanggal 16 Oktober 1963\\
PRESIDEN REPUBLIK INDONESIA,\\
PYM Ir. SUKARNO
\\

Diundangkan Di Jakarta,\\
Pada Tanggal 16 Oktober 1963\\
SEKRETARIS NEGARA REPUBLIK INDONESIA,\\
Ttd. MOHD. ICHSAN\\
LEMBARAN NEGARA REPUBLIK INDONESIA TAHUN 1963 NOMOR 101
\end{document}
